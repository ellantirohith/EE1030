\let\negmedspace\undefined
\let\negthickspace\undefined
\documentclass[journal,12pt,onecolumn]{IEEEtran}
\usepackage{cite}
\usepackage{amsmath,amssymb,amsfonts,amsthm}
\usepackage{amsmath}
\usepackage{algorithmic}
\usepackage{graphicx}
\usepackage{textcomp}
\usepackage{xcolor}
\usepackage{txfonts}
\usepackage{listings}
\usepackage{multicol}
\usepackage{enumitem}
\usepackage{mathtools}
\usepackage{gensymb}
\usepackage{comment}
\usepackage[breaklinks=true]{hyperref}
\usepackage{tkz-euclide} 
\usepackage{listings}
\usepackage{gvv}                                        
\usepackage[latin1]{inputenc}                                
\usepackage{color}                                            
\usepackage{array}                                            
\usepackage{longtable}                                       
\usepackage{calc}                                             
\usepackage{multirow}                                         
\usepackage{hhline}                                           
\usepackage{ifthen}                                           
\usepackage{lscape}
\usepackage{tabularx}
\usepackage{array}
\usepackage{float}
\usepackage{circuitikz}

\newtheorem{theorem}{Theorem}[section]
\newtheorem{problem}{Problem}
\newtheorem{proposition}{Proposition}[section]
\newtheorem{lemma}{Lemma}[section]
\newtheorem{corollary}[theorem]{Corollary}
\newtheorem{example}{Example}[section]
\newtheorem{definition}[problem]{Definition}
\newcommand{\BEQA}{\begin{eqnarray}}
\newcommand{\EEQA}{\end{eqnarray}}
\newcommand{\define}{\stackrel{\triangle}{=}}
\theoremstyle{remark}
\newtheorem{rem}{Remark}

\begin{document}

\bibliographystyle{IEEEtran}
\vspace{3cm}

\title{2007 PH 69-85}
\author{EE24BTECH11020 -  Ellanti Rohith}
\maketitle

\renewcommand{\thefigure}{\theenumi}
\renewcommand{\thetable}{\theenumi}


\begin{enumerate}
    


\item In the circuit shown, the voltage at test point P is 12V and the voltage between gate and source is -2V. The value of $R$ (in k$\Omega$) is  \hfill{[GATE 2007]}\\
\begin{center}
    

\begin{circuitikz} \draw
(0,3) -- (0,-0.5) -- (0.5,-0.5)
to[R=$R$] (2,-0.5) --(3,-0.5)
(0,2) --(0.5,2) 
to[R=2k$\Omega$] (2,2)--(2.5,2)--(2.5,3) ;
\draw(3,-0.5) -- (4.1,-0.5)  to[R=42k$\Omega$] (5.34,-0.5) -- (6,-0.5)-- (6,2)--(5.75,2) to[R=4k$\Omega$] (4.5,2) --(4,2) ;
\draw (6,-0.5) -- (6,-1) node[ground] {};
\draw (2.5,2)--(3.1,2)--(3.1,1.25)--(3.65,1.25)--(3.65,2)--(4,2);
\filldraw[black] (2.5,3) circle (2pt); 
    \node[above] at (2.5,3) {$P$}; 
   \fill[black] (3.37,1.25) -- (3.25,1) -- (3.49,1) -- cycle;
    \draw(3.37,1.25)--(3.37,-0.5);
      \draw[thick] (3.37,1.7)ellipse [x radius=14pt, y radius=16pt];
    \filldraw[black] (0,3) circle (2pt); 
    \node[above] at (0,3) {$V_{DD}=16$}; 
\end{circuitikz}

   
\end{center}

\begin{multicols}{4}
 \begin{enumerate}
     \item 42 
    \item 48 
    \item 56 
    \item 70
 \end{enumerate}
\end{multicols}   


\item  When an input voltage $V_i$, of the form shown, is applied to the circuit given below, the output voltage $V_o$ is of the form   \hfill{[GATE 2007]}


 \begin{circuitikz}
    \draw
    (0,0)--(1,0) to[R=R] (2.5,0)--(3,0)--(3,-1);
    \draw[line width=2pt](2.6,-1)--(3.4,-1);
   \fill[black] (3,-1) -- (2.6,-1.25) -- (3.4,-1.25) -- cycle;
   \draw(3,-1.25)--(3,-1.5);
    \draw[line width=2pt](2.6,-1.5)--(3.4,-1.5); \node[above] at (2.3,-2.3) {$3V$};  
\draw[line width=2pt](2.8,-1.75)--(3.2,-1.75);
\draw(3,-1.75)--(3,-2.3);
\draw(3,0)--(4,0);
\draw(0,-2.3)--(4,-2.3);
\draw[<-] (0,0) -- (0,-0.5);
\draw[->] (0,-1.8) -- (0,-2.3); 
\node at (0, -1.1) {$v_i$};
\draw[<-] (4,0) -- (4,-0.5);
\draw[->] (4,-1.8) -- (4,-2.3); 
\node at (4, -1.1) {$v_o$};

  \draw[dashed] (-5,0) -- (-1.5,0);
  \node at (-5.7,0) {+12 V};
  \draw[dashed] (-5,-2.3) -- (-1.5,-2.3);
  \node at (-5.7,-2.3) {-12 V};
  \draw(-5.25,-1.15)--(-1.5,-1.15);\draw[->] (-1.5,-1.15) -- (-1.25,-1.15); 
  
 
  \draw[thick] (-4.5,-1.15) -- (-4,0) -- (-3.5,-1.15) -- (-3,-2.3) -- (-2.5,-1.15);
\end{circuitikz}
\\
\begin{multicols}{2}
    \begin{enumerate}
      \item
    \begin{circuitikz}
   \node at (-9,0) {12V};
      \node at (-9,-1) {0V};
      \draw (-8.5,-1) -- (-5,-1); 
    \draw[->] (-5,-1) -- (-4,-1);
    \draw[thick] (-8.5,-1) -- (-8,-0) -- (-7.5,-1) -- (-5.5,-1) -- (-5,-0) -- (-4.5,-1);
\end{circuitikz}
     \\
\item
\begin{circuitikz}

    \node at (-10,0) {12V};
    \node at (-10,-1.5) {0V}; 
    \node at (-10,-1) {3V};
       \draw (-9, -1.5) -- (-5, -1.5); 
    \draw[->] (-5, -1.5) -- (-4.75, -1.5); 
    \draw[thick] (-9,-1) -- (-8.5,-1) -- (-8,0) -- (-7.5,-1) -- (-7,-1) -- (-6.5,0) -- (-6,-1) -- (-5.5,-1);
   \end{circuitikz}
  \item
    \begin{circuitikz}
        \node at (-9, -1) {-12V};
        \node at (-9, 0) {0V};
        \draw (-8.5, 0) -- (-5, 0);
        \draw[->] (-5, 0) -- (-4, 0);
\draw[thick] (-8.5,0) -- (-8,-1) -- (-7.5,0) -- (-5.5,0) -- (-5,-1) -- (-4.5,0);        
    
    \end{circuitikz}
\\ 
\item
\begin{circuitikz}
    \node at (-10,0) {12V};
    \node at (-10,-1.5) {0V}; 
    \node at (-10,-1) {2.3V};
    \draw (-9, -1.5) -- (-5, -1.5); 
    \draw[->] (-5, -1.5) -- (-4.75, -1.5); 
    \draw[thick] (-9,-1) -- (-8.5,-1) -- (-8,0) -- (-7.5,-1) -- (-7,-1) -- (-6.5,0) -- (-6,-1) -- (-5.5,-1);
 \end{circuitikz}
  \end{enumerate}
\end{multicols}


\textbf{Common Data for Questions 71, 72, 73:}

A particle of mass $m$ is confined in the ground state of a one-dimensional box, extending from $x = -2L$ to $x = +2L$. The wavefunction of the particle in this state is  
\begin{align*}
    \psi(x) = \psi_0 \cos \brak{ \frac{\pi x}{4L}}
\end{align*}
where $\psi_0$ is a constant.\hfill{[GATE 2007]}


    \item  The normalization factor $\psi_0$ of this wavefunction is
    \begin{multicols}{4}
    \begin{enumerate}
        \item $\frac{\sqrt{2}}{L}$
        \item $\frac{1}{\sqrt{4L}}$
        \item $\frac{1}{\sqrt{2L}}$
        \item $\frac{1}{\sqrt{L}}$
    \end{enumerate}
    \end{multicols}

    \item  The energy eigenvalue corresponding to this state is
    \begin{multicols}{4}
    \begin{enumerate}
        \item $\frac{\hbar^2 \pi^2}{2mL^2}$
        \item $\frac{\hbar^2 \pi^2}{4mL^2}$
        \item $\frac{\hbar^2 \pi^2}{16mL^2}$
        \item $\frac{\hbar^2 \pi^2}{32mL^2}$
    \end{enumerate}
    \end{multicols}

    \item  The expectation value of $p^2$ ($p$ is the momentum operator) in this state is
    \begin{multicols}{2}
    \begin{enumerate}
        \item $0$\\
        \item $\frac{\hbar^2 \pi^2}{32L^2}$
\\        \item $\frac{\hbar^2 \pi^2}{16L^2}$
        \item $\frac{\hbar^2 \pi^2}{8L^2}$
    \end{enumerate}
    \end{multicols}


\textbf{Common Data for Questions 74, 75:}

The Fresnel relations between the amplitudes of incident and reflected electromagnetic waves at an interface between air and a dielectric of refractive index $\mu_r$ are   
\begin{align*}
    \frac{E_{\parallel}^{\text{reflected}}}{E_{\parallel}^{\text{incident}}} = \frac{\cos r - \mu \cos i}{\cos r + \mu \cos i}, \quad \frac{E_{\perp}^{\text{reflected}}}{E_{\perp}^{\text{incident}}} = \frac{\mu \cos r - \cos i}{\mu \cos r + \cos i}
\end{align*}
The subscripts $\parallel$ and $\perp$ refer to polarization, parallel and normal to the plane of incidence respectively. Here, $i$ and $r$ are the angles of incidence and refraction respectively.\hfill{[GATE 2007]}


    \item  The condition for the reflected ray to be completely polarized is
    \begin{multicols}{2}
    \begin{enumerate}
        \item $\mu \cos i = \cos r$
        \item $\cos i = \mu \cos r$
        \item $\mu \cos i = - \cos r$
        \item $\cos i = - \mu \cos r$
    \end{enumerate}
    \end{multicols}

    \item  For normal incidence at an air-glass interface with $\mu = 1.5$, the fraction of energy reflected is given by
    \begin{multicols}{4}
    \begin{enumerate}
        \item 0.40
        \item 0.20
        \item 0.16
        \item 0.04
    \end{enumerate}
    \end{multicols}



\textbf{Statement for Linked Answer Questions 76 \& 77:}

In the laboratory frame, a particle $P$ of rest mass $m_0$ is moving in the positive $x$ direction with a speed of $\frac{5c}{19}$. It approaches an identical particle $Q$, moving in the negative $x$ direction with a speed of $\frac{2c}{5}$.  \hfill{[GATE 2007]}

    \item  The speed of the particle $P$ in the rest frame of the particle $Q$ is
    \begin{multicols}{4}
    \begin{enumerate}
        \item $\frac{7c}{95}$
        \item $\frac{13c}{85}$
        \item $\frac{3c}{5}$
        \item $\frac{63c}{95}$
    \end{enumerate}
    \end{multicols}

    \item  The energy of the particle $P$ in the rest frame of the particle $Q$ is
    \begin{multicols}{4}
    \begin{enumerate}
        \item $\frac{1}{2} m_0 c^2$
        \item $\frac{5}{4} m_0 c^2$
        \item $\frac{19}{13} m_0 c^2$
        \item $\frac{11}{9} m_0 c^2$
    \end{enumerate}
    \end{multicols}


\bigskip

\textbf{Statement for Linked Answer Questions 78 \& 79:}

The atomic density of a solid is $5.85 \times 10^{28} \text{m}^{-3}$. Its electrical resistivity is $1.6 \times 10^{-8} \, \Omega \, \text{m}$. Assume that electrical conduction is described by the Drude model (classical theory), and that each atom contributes one conduction electron.  \hfill{[GATE 2007]}


    \item The drift mobility (in $\text{m}^2 \text{V}^{-1} \text{s}^{-1}$) of the conduction electrons is
    \begin{multicols}{4}
    \begin{enumerate}
        \item $6.67 \times 10^{-3}$
        \item $6.67 \times 10^{-6}$
        \item $7.63 \times 10^{-3}$
        \item $7.63 \times 10^{-6}$
    \end{enumerate}
    \end{multicols}

    \item The relaxation time (mean free time), in seconds, of the conduction electrons is
    \begin{multicols}{2}
    \begin{enumerate}
        \item $3.98 \times 10^{-15}$
        \item $3.79 \times 10^{-14}$
        \item $2.84 \times 10^{-12}$
        \item $2.64 \times 10^{-11}$
    \end{enumerate}
    \end{multicols}


\bigskip

\textbf{Statement for Linked Answer Questions 80 \& 81:}

A sphere of radius $R$ carries a polarization $\vec{P} = k\vec{r}$, where $k$ is a constant and $\vec{r}$ is measured from the center of the sphere.   \hfill{[GATE 2007]}


    \item  The bound surface and volume charge densities are given, respectively, by
    \begin{multicols}{2}
    \begin{enumerate}
        \item $-k|r|$ and $3k$
        \item $k|r|$ and $-3k$
        \item $k|r|$ and $-4\pi k R$
        \item $-k|r|$ and $4\pi k R$
    \end{enumerate}
    \end{multicols}

    \item  The electric field $\vec{E}$ at a point $\vec{r}$ outside the sphere is given by
    \begin{multicols}{2}
    \begin{enumerate}
        \item $\vec{E} = 0$\\
        \item $\vec{E} = \frac{kR(R^2 - r^2)}{6 \varepsilon_0 r^3} \hat{r}$
        \item $\vec{E} = \frac{kR(R^2 - r^2)}{\varepsilon_0 r^3} \hat{r}$\\
        \item $\vec{E} = \frac{3k(r - R)}{4\pi \varepsilon_0 r^4} \hat{r}$
    \end{enumerate}
    \end{multicols}



\textbf{Statement for Linked Answer Questions 82 \& 83:}

An ensemble of quantum harmonic oscillators is kept at a finite temperature $T = \frac{1}{k_B\beta$}.  \hfill{[GATE 2007]}


    \item  The partition function of a single oscillator with energy levels $\left(n + \frac{1}{2}\right) \hbar \omega$ is given by
    \begin{multicols}{2}
    \begin{enumerate}
        \item $Z = \frac{e^{-\beta \hbar \omega / 2}}{1 - e^{-\beta \hbar \omega}}$\\
        \item $Z = \frac{e^{-\beta \hbar \omega / 2}}{1 + e^{-\beta \hbar \omega}}$
        \item $Z = \frac{1}{1 - e^{-\beta \hbar \omega}}$\\
        \item $Z = \frac{1}{1 + e^{-\beta \hbar \omega}}$
    \end{enumerate}
    \end{multicols}

    \item  The average number of energy quanta of the oscillators is given by
    \begin{multicols}{2}
    \begin{enumerate}
        \item $\langle n \rangle = \frac{1}{e^{\beta \hbar \omega} - 1}$\\
        \item $\langle n \rangle = \frac{e^{-\beta \hbar \omega}}{e^{\beta \hbar \omega} - 1}$
        \item $\langle n \rangle = \frac{1}{e^{\beta \hbar \omega} + 1}$\\
        \item $\langle n \rangle = \frac{e^{-\beta \hbar \omega}}{e^{\beta \hbar \omega} + 1}$
    \end{enumerate}
    \end{multicols}


\bigskip

\textbf{Statement for Linked Answer Questions 84 \& 85:}

A $16 \, \mu A$ beam of alpha particles, having cross-sectional area $10^{-4} \, \text{m}^2$, is incident on a rhodium target of thickness $1 \, \mu \text{m}$. This produces neutrons through the reaction:
\[
\alpha + {}^{100}Rh \rightarrow {}^{101}Pd + 3n
\] \hfill{[GATE 2007]}


    \item  The number of alpha particles hitting the target per second is
    \begin{multicols}{4}
    \begin{enumerate}
        \item $0.5 \times 10^{14}$
        \item $1.0 \times 10^{14}$
        \item $2.0 \times 10^{20}$
        \item $4.0 \times 10^{20}$
    \end{enumerate}
    \end{multicols}

    \item  The neutrons are observed at the rate of $1.806 \times 10^8 \, \text{s}^{-1}$. If the density of rhodium is approximated as $10^4 \, \text{kg} \, \text{m}^{-3}$, the cross-section for the reaction (in barns) is
    \begin{multicols}{4}
    \begin{enumerate}
        \item 0.1
        \item 0.2
        \item 0.4
        \item 0.8
    \end{enumerate}
    \end{multicols}
\end{enumerate}


\end{document}

