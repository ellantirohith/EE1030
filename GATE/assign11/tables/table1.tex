\iffalse
\title{assignment}
\author{EE24BTECH11016}
\section{mcq-single}
\fi



    \item Let the six numbers $a_{1},a_{2},a_{3},a_{4},a_{5},a_{6}$ be in $A.P$ and $a_{1}+a_{2}=10$. If the mean of these six numbers is $ \frac{19}{2} $ and their variance is $\sigma^{2}$, then 8$\sigma^{2}$ is equal to 
    \hfill{Jan 2023}\begin{enumerate}
        \item 220
        \item 210
        \item 200
        \item 105
    \end{enumerate}
    \item Let $f(x)$ be a function such that $f(x+y)=f(x)\cdot f(y)$ for all x, y $\in \mathbb{N}$ . If  $ f(1)=3 $  and  $ \sum_{k=1}^{n} f(k)= 3279$, then the value of $n$ is
    \hfill{Jan 2023}\begin{enumerate}
        \item 6
        \item 8
        \item 7
        \item 9
    \end{enumerate} 
    \item The number of real solutions of the equations $3\left( x^{2}+\frac{1}{x^{2}} \right)-2\left( x+\frac{1}{x} \right)+5=0$, is
    \hfill{Jan 2023}\begin{enumerate}
        
    \item 4
    \item 0
    \item 3
    \item 2
    \end{enumerate}
    \item If $f(x)=\frac{2^{2x}}{2^{2x}+2}, x \in \mathbb{R}$, \\
    then $f\left( \frac{1}{2023} \right)+f\left( \frac{2}{2023} \right)+\cdots\cdots+f\left( \frac{2022}{2023} \right)$ is equal to
    \hfill{Jan 2023}\begin{enumerate}
        \item 2011
        \item 1010
        \item 2010
        \item 1011
    \end{enumerate}
    \item If $ f(x)=x^{3}-x^{2}f'(1)+xf''(2)-f'''(3), x \in \mathbb{R}, $ then 
    \hfill{Jan 2023}\begin{enumerate}
        \item $ 3f(1) + f(2) = f(3) $
        \item $ f(3)-f(2) = f(1) $
        \item $ 2f(0)-f(1) + f(3) = f(2) $
        \item $ f(1) + f(2) + f(3) = f(0) $
    \end{enumerate}
    \item The number of integers, greater than $7000$ that can
be formed, using the digits $3, 5, 6, 7, 8$ without
repetition, is
\hfill{Jan 2023}\begin{enumerate}
    \item 120
    \item 168
    \item 220
    \item 48
\end{enumerate}
    \item If the system of equations \\
 $ x + 2y + 3z = 3$ \\
$ 4x + 3y-4z = 4$ \\
 $8x + 4y-\lambda z = 9 + \mu $ \\ 
 has infinitely many solutions, then the ordered pair
$(\lambda, \mu)$ is equal to
\hfill{Jan 2023}\begin{enumerate}
    \item $ \left( \frac{72}{5},\frac{21}{5} \right)$
    \item $ \left( \frac{-72}{5},\frac{-21}{5} \right)$
    \item $ \left( \frac{72}{5},\frac{-21}{5} \right)$
    \item $ \left( \frac{-72}{5},\frac{21}{5} \right)$
\end{enumerate}
\item The value of $\left( \frac{1+\sin{\frac{2\pi}{9}}+i\cos{\frac{2\pi}{9}}}{1+\sin{\frac{2\pi}{9}}-i\cos{\frac{2\pi}{9}}} \right)^{3}$ is
\hfill{Jan 2023}\begin{enumerate}
    \item $\frac{-1}{2} (1-i\sqrt{3})$
    \item $ \frac{1}{2} (1-i\sqrt{3})$
    \item $ \frac{-1}{2} (\sqrt{3}-i) $
    \item $ \frac{1}{2} (\sqrt{3}+i) $
\end{enumerate}
\item The equations of the sides AB and AC of a triangle
ABC are \\
$(\lambda + 1) x + \lambda y = 4 $ and $ \lambda x + (1-\lambda) y + \lambda = 0$
 respectively. Its vertex A is on the y-axis and its
orthocentre is $(1, 2)$. The length of the tangent
from the point C to the part of the parabola $y^{2}
 = 6x$
in the first quadrant is 
\hfill{Jan 2023}\begin{enumerate}
    \item $\sqrt{6}$
    \item $2\sqrt{2}$
    \item 2
    \item 4
\end{enumerate}
\item The set of all values of $a$ for which \\
$\lim_{x \to a} ([x-5]-[2x+2])=0$, where $[x]$ denotes the greatest integer less than or equal to $\infty$ is equal to 
\hfill{Jan 2023}\begin{enumerate}
    \item $(-7.5, -6.5)$
    \item $(-7.5, -6.5] $
    \item $[-7.5, -6.5] $
    \item $[-7.5, -6.5) $
\end{enumerate}
\item If $ ({}^{30}C_{1})^{2} + 2({}^{30}C_{2})^{2} + 3({}^{30}C_{3})^{2} +......+ 30({}^{30}C_{30})^{2} = \frac{\alpha60!}{(30!)^{2}} $, then $\alpha$ is equal to 
\hfill{Jan 2023}\begin{enumerate}
    \item 30
    \item 60
    \item 15
    \item 10
\end{enumerate}
\item Let the plane containing the line of intersection
of the planes \\
$P1: x + (\lambda + 4)y + z = 1$ and \\
$P2 : 2x + y + z = 2$ pass through the points $(0, 1, 0)$
and $(1, 0, 1)$. Then the distance of the point
$(2\lambda, \lambda, -\lambda)$ from the plane $P2$ is
\hfill{Jan 2023}\begin{enumerate}
    \item $ 5\sqrt{6}$
    \item $ 4\sqrt{6}$
    \item $ 2\sqrt{6}$
    \item $ 3\sqrt{6}$
\end{enumerate}
\item $\overrightarrow{\alpha}=4\hat{i}+3\hat{j}+5\hat{k}$ and $ \overrightarrow{\beta}=\hat{i}+2\hat{j}-4\hat{k}$. Let  $\overrightarrow{\beta}_{1}$ be parallel to $\overrightarrow{\alpha}$ and $\overrightarrow{\beta}_{2}$ be perpendicular to $\overrightarrow{\alpha}$. If $\overrightarrow{\beta}=\overrightarrow{\beta}_{1}+\overrightarrow{\beta}_{2}$, then the value of $5\overrightarrow{\beta}_{2}\cdot(\hat{i}+\hat{j}+\hat{k})$ is
\hfill{Jan 2023}\begin{enumerate}
    \item 6
    \item 11
    \item 7
    \item 9
\end{enumerate}
\item The locus of the mid points of the chords of the
circle $C_{1}:(x-4)^{2}+(y-5)^{2}=4$ which subtend an angle $\theta_{i}$ at the center of the circle $C_{1}$, is a circle of radius $ r_{i}$. If $ \theta_{1}=\frac{\pi}{3},\theta_{3}=\frac{2\pi}{3} $ and $ r_{1}^{2}=r_{2}^{2}+r_{3}^{2}$ then $\theta_{2}$ is equal to  
\hfill{Jan 2023}\begin{enumerate}
    \item $\frac{\pi}{4}$
    \item $\frac{3\pi}{4}$
    \item $\frac{\pi}{6}$
    \item $\frac{\pi}{2}$

\end{enumerate}
\item If the foot of the perpendicular drawn from $(1, 9,
7)$ to the line passing through the point $(3, 2, 1)$ and
parallel to the planes $x + 2y + z = 0$ and $3y-z = 3$
is $(\alpha, \beta, \gamma)$, then $\alpha + \beta + \gamma$ is equal to
\hfill{Jan 2023}\begin{enumerate}
    \item -1
    \item 3
    \item 1
    \item 5
\end{enumerate}

