\let\negmedspace\undefined
\let\negthickspace\undefined
\documentclass[journal,12pt,onecolumn]{IEEEtran}
\usepackage{cite}
\usepackage{amsmath,amssymb,amsfonts,amsthm}
\usepackage{amsmath}
\usepackage{algorithmic}
\usepackage{graphicx}
\usepackage{textcomp}
\usepackage{xcolor}
\usepackage{txfonts}
\usepackage{listings}
\usepackage{multicol}
\usepackage{enumitem}
\usepackage{mathtools}
\usepackage{gensymb}
\usepackage{comment}
\usepackage[breaklinks=true]{hyperref}
\usepackage{tkz-euclide} 
\usepackage{listings}
\usepackage{gvv}                                        
\usepackage[latin1]{inputenc}                                
\usepackage{color}                                            
\usepackage{array}                                            
\usepackage{longtable}                                       
\usepackage{calc}                                             
\usepackage{multirow}                                         
\usepackage{hhline}                                           
\usepackage{ifthen}                                           
\usepackage{lscape}
\usepackage{tabularx}
\usepackage{array}
\usepackage{float}
\usepackage{circuitikz}

\newtheorem{theorem}{Theorem}[section]
\newtheorem{problem}{Problem}
\newtheorem{proposition}{Proposition}[section]
\newtheorem{lemma}{Lemma}[section]
\newtheorem{corollary}[theorem]{Corollary}
\newtheorem{example}{Example}[section]
\newtheorem{definition}[problem]{Definition}
\newcommand{\BEQA}{\begin{eqnarray}}
\newcommand{\EEQA}{\end{eqnarray}}
\newcommand{\define}{\stackrel{\triangle}{=}}
\theoremstyle{remark}
\newtheorem{rem}{Remark}

\begin{document}

\bibliographystyle{IEEEtran}
\vspace{3cm}

\title{2008 XE}
\author{EE24BTECH11020 -  Ellanti Rohith}
\maketitle

\renewcommand{\thefigure}{\theenumi}
\renewcommand{\thetable}{\theenumi}
\begin{enumerate}

\item The solution of the first-order differential equation $\brak{0 \leq x < 1}$
\begin{align*}
    \frac{dy}{dx} - y^2 = 0,\text{ } y(0) = 1 \text{ is:}
\end{align*} 
\hfill{[GATE 2008]}
\begin{multicols}{4}
\begin{enumerate}
    \item $\frac{1}{1 + x} $ 
    \item $ \frac{1}{1 - x} $
    \item $ \frac{2}{2 + x} $
\    \item $ \frac{x^3}{3} + 1 $
\end{enumerate}
\end{multicols}

\item For the initial value problem 
\begin{align*}
    \frac{dy}{dx} + y = 0 ,  \text{ }y(0) = 1 , 
\end{align*} $ y_1 $  is the computed value of $ y $ at $ x = 0.2 $ obtained by using Euler's method with step size $ h = 0.1 $. Then,
\hfill{[GATE 2008]}
\begin{multicols}{2}
    \begin{enumerate}
    \item $ y_1 < e^{-0.2} $
    \item $ e^{-0.2} < y_1 < 1 $
    \item $ 1 < y_1 $
    \item $ y_1 = e^{-0.2} $
\end{enumerate}
\end{multicols}


\item  Consider the initial value problem \begin{align*}
    \frac{dy}{dx} = y + x , \text{ }  y(0) = 2\text{ .} 
\end{align*}
The value of $ y_1 $ obtained using the fourth order Runge-Kutta method with step size $ h = 0.1 $ is\hfill{[GATE 2008]}
\begin{multicols}{4}
    \begin{enumerate}
    \item $ 2.20000 $
    \item $ 2.21500 $
    \item $ 2.21551 $
    \item $ 2.21576 $
\end{enumerate}
\end{multicols}


\item The following table gives a function $f\brak{x}\text{ vs }x$:



\[
\begin{tabular}{|c|c|c|c|c|c|}
\hline
$x$ & 0 & 1 & 2 & 3 & 4 \\
\hline
$f(x)$ & 1.0 & 3.7 & 6.5 & 9.3 & 12.1 \\
\hline
\end{tabular}
\]

The best fit of a straight line for the above data points, using a least square error method is:
\\ \text{ }\hfill{[GATE 2008]}


\begin{multicols}{4}
    \begin{enumerate}
    \item $2.75x + 0.55$
    \item $2.80x + 0.80$
    \item $3.10x + 0.85$
    \item $2.78x + 0.96$
\end{enumerate}
\end{multicols}



\item Consider the following part of a Fortran 90 function:

\begin{verbatim}
INTEGER FUNCTION RESULT(X)
    INTEGER:: X
    VALUE = 1
    DO
        IF (X == 0) EXIT
        TERM = MOD (X,10)
        VALUE = VALUE*TERM
        X = X/10
    END DO
    RESULT = VALUE
END FUNCTION RESULT
\end{verbatim}

If the above function is called with an integer X = 123, the value returned by the function will be: \\ \text{ }\hfill{[GATE 2008]}
\begin{multicols}{4}
\begin{enumerate}
    \item 0
    \item 6
    \item 9
    \item 321
\end{enumerate}
\end{multicols}
\item A portion of a Fortran 90 program is reproduced below:

\begin{verbatim}
    PROGRAM CHECK_CYCLE
        DO I = 1, 10, 2
            IF (MOD(I, 3) == 0) CYCLE
            PRINT *, I
        END DO
    END PROGRAM CHECK_CYCLE
\end{verbatim}

The result displayed by the program is:\hfill{[GATE 2008]}

\begin{multicols}{4}
\begin{enumerate}
    \item 1 \\
          5 \\
          7
          \columnbreak
    \item 1 \\
          3 \\
          5
\columnbreak    \item 1 \\
          3 \\
          7
\columnbreak    \item 3 \\
          5 \\
          7
\end{enumerate}
\end{multicols}

\item (P), (Q), (R) and (S) are separate segments of Fortran 90 code.
\begin{center}
    \begin{verbatim}
(P) IF (A > B) P = Q

(Q) SUBROUTINE SWAP(A, B)
    INTEGER, INTENT(IN) :: A, B
    TEMP = A
    A = B
    B = TEMP
    END SUBROUTINE SWAP

(R) IF (A /= B) X = Y - Z
    ELSE
    X = Y + Z
    ENDIF

(S) DO I = 1, N, 3
    C(I) = A(I) + B(I)
    END DO
   \end{verbatim}
  
\end{center}
    




Which segments have syntax errors?\hfill{[GATE 2008]}
\begin{multicols}{4}
\begin{enumerate}
    \item P, Q
    \item Q, R
    \item R, S
    \item P, S
\end{enumerate}
\end{multicols}
    
     \item A Fortran-90 subroutine for the Gauss-Seidel Method to solve a set of \( N \) simultaneous equations \([A][X]=[C]\) is given below:

\begin{verbatim}
    SUBROUTINE SIEDEL(A, C, X, N, IMAX)
    REAL :: SUM
    REAL, DIMENSION(N,N) :: A
    REAL, DIMENSION(N) :: C, X
    DO K = 1, IMAX
        DO I = 1, N
            SUM = 0.0
            DO J = 1, N
                IF (I /= J) THEN
                    SUM = SUM + A(I,J)*X(J)
                ENDIF
            ENDDO
            ******
        ENDDO
    ENDDO
    END SUBROUTINE SIEDEL
\end{verbatim}

The missing statement in the program, indicated by `******`, is:\hfill{[GATE 2008]}

\begin{multicols}{4}
\begin{enumerate}
    \item $ X(I) = C(I) + SUM $
    \item $ X(I) = C(I) - SUM $
    \item $ X(I) = \frac{C(I) + SUM}{A(I,I)} $
    \item $ X(I) = \frac{C(I) - SUM}{A(I,I)} $
\end{enumerate}
\end{multicols}

\item What is the result of the following C program?

\begin{verbatim}
    int main()
    {
        int  i, sum=0;
        for (i = 0; i < 25; i++) {
            if (i > 10) continue;
            sum += i;
        }
        printf("%d\n",sum);
        return 1;
    }
\end{verbatim}
\hfill{[GATE 2008]}

\begin{multicols}{4}
\begin{enumerate}
    \item $ 25 $
    \item $ 45 $
    \item $ 55 $
    \item $ 325 $
\end{enumerate}
\end{multicols}

\item Consider the following C code:

\begin{verbatim}
    int x = 1, y = 5, z;
    z = x++<<--y;
\end{verbatim}

The values of $ x $, $ y$, and $ z$ after the execution are:
\hfill{[GATE 2008]}
\begin{multicols}{4}
\begin{enumerate}
    \item $ 2, 4, 16 $
    \item $ 2, 4, 32 $
    \item $ 2, 4, 64 $
    \item $ 1, 5, 32 $
\end{enumerate}
\end{multicols}

\item A two-dimensional array is declared as \texttt{int num[3][3]}. Then the result of the expression \texttt{*(num+1)} is:
\hfill{[GATE 2008]}

\begin{enumerate}
    \item The value of \texttt{num[1][0]}
    \item The value of \texttt{num[0][1]}
    \item The address of \texttt{num[1][0]}
    \item The address of \texttt{num[0][1]}
\end{enumerate}

\item A C function named \texttt{func} is defined as follows:
\begin{verbatim}
    int func(int a, int b)  {
        if (  (a == 1)||(b == 0)||(a == b) ) return 1;
        return func(a-1,b) + func(a-1,b-1);
    }
\end{verbatim}


What is the result of \texttt{func(4, 2)}?\hfill{[GATE 2008]}

\begin{multicols}{4}
\begin{enumerate}
    \item 12
    \item 6
    \item 3
    \item 1
\end{enumerate}
\end{multicols}
 \textbf{Common Data for Questions 29 and 30}
The following table gives the values of a function $f(x)$ at three discrete points.

\[
\begin{array}{|c|c|c|c|}
\hline
x & 0.5 & 0.6 & 0.7 \\
\hline
f(x) & 0.4794 & 0.5646 & 0.6442 \\
\hline
\end{array}
\]
\hfill{[GATE 2008]}
\item The value of $f'(x)$ at $x = 0.5$ accurate up to two decimal places, is

\begin{multicols}{4}
\begin{enumerate}
    \item 0.82
    \item 0.85
    \item 0.88
    \item 0.91
\end{enumerate}
\end{multicols}
\item The value of $f(x)$ at $x = 0.55$ obtained using Newton's interpolation formula, is

\begin{multicols}{4}
\begin{enumerate}
    \item 0.5626
    \item 0.5227
    \item 0.4847
    \item 0.4749
\end{enumerate}
\end{multicols}

\textbf{Linked Answer Questions: Q.31 to Q.34 carry two marks each.}

A modified Newton-Raphson method is used to find the roots of an equation $f(x) = 0$ which has multiple zeros at some point $x = p$ in the interval $[a, b]$. If the multiplicity $M$ of the root is known in advance, an iterative procedure for determining $p_{k+1}$ is given by
\begin{align*}
  p_{k+1} = p_k - M \frac{f(p_k)}{f'(p_k)} \quad \text{for} \quad k = 0, 1, 2, \dots  
\end{align*}\hfill{[GATE 2008]}

\item The equation $f(x) = x^3 - 1.8x^2 - 1.35x + 2.7 = 0$ is known to have a multiple root in the interval $[1, 2]$. Starting with an initial guess $x_0 = 1.0$ in the modified Newton-Raphson method, the root, correct up to three decimal places, is

\begin{multicols}{4}
\begin{enumerate}
    \item 1.500
    \item 1.200
    \item 1.578
    \item 1.495
\end{enumerate}
\end{multicols}

\item The root of the derivative of $f(x)$ in the same interval is

\begin{multicols}{4}
\begin{enumerate}
    \item 1.500
    \item 1.200
    \item 1.578
    \item 1.495
\end{enumerate}
\end{multicols}

\textbf{Statement for Linked Answer Questions 33 and 34:}

The values of a function $f(x)$ at four discrete points are as follows:

\[
\begin{array}{|c|c|c|c|c|}
\hline
x & 0 & 1 & 3 & 4 \\
\hline
f(x) & -12 & 0 & 6 & 12 \\
\hline
\end{array}
\]
\item The function may be represented by a polynomial $P(x) = (x - a)R(x)$, where $R(x)$ is a polynomial of degree 2, obtained by Lagrange's interpolation and $a$ is a real constant. The polynomial $R(x)$ is

\begin{multicols}{4}
\begin{enumerate}
    \item $x^2 + 6x + 12$
    \item $x^2 + 6x - 12$
    \item $x^2 - 6x - 12$
    \item $x^2 - 6x + 12$
\end{enumerate}
\end{multicols}
\end{enumerate}


\end{document}


