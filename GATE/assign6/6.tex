\let\negmedspace\undefined
\let\negthickspace\undefined
\documentclass[journal,12pt,onecolumn]{IEEEtran}
\usepackage{cite}
\usepackage{amsmath,amssymb,amsfonts,amsthm}
\usepackage{amsmath}
\usepackage{algorithmic}
\usepackage{graphicx}
\usepackage{textcomp}
\usepackage{xcolor}
\usepackage{txfonts}
\usepackage{listings}
\usepackage{multicol}
\usepackage{enumitem}
\usepackage{mathtools}
\usepackage{gensymb}
\usepackage{comment}
\usepackage[breaklinks=true]{hyperref}
\usepackage{tkz-euclide} 
\usepackage{listings}
\usepackage{gvv}                                        
\usepackage[latin1]{inputenc}                                
\usepackage{color}                                            
\usepackage{array}                                            
\usepackage{longtable}                                       
\usepackage{calc}                                             
\usepackage{multirow}                                         
\usepackage{hhline}                                           
\usepackage{ifthen}                                           
\usepackage{lscape}
\usepackage{tabularx}
\usepackage{array}
\usepackage{float}
\usepackage{tikz}
\usepackage{multicol}
\usepackage{circuitikz}
\usepackage{amsmath}
\usetikzlibrary{patterns}
\newtheorem{theorem}{Theorem}[section]
\newtheorem{problem}{Problem}
\newtheorem{proposition}{Proposition}[section]
\newtheorem{lemma}{Lemma}[section]
\newtheorem{corollary}[theorem]{Corollary}
\newtheorem{example}{Example}[section]
\newtheorem{definition}[problem]{Definition}
\newcommand{\BEQA}{\begin{eqnarray}}
\newcommand{\EEQA}{\end{eqnarray}}
\newcommand{\define}{\stackrel{\triangle}{=}}
\theoremstyle{remark}
\newtheorem{rem}{Remark}

\begin{document}

\bibliographystyle{IEEEtran}
\vspace{3cm}

\title{2017-EE}
\author{EE24BTECH11020 -  Ellanti Rohith}
\maketitle

\renewcommand{\thefigure}{\theenumi}
\renewcommand{\thetable}{\theenumi}

\begin{enumerate}
\item Consider the differential equation 
\begin{align*}
\brak{t^2 - 81} \frac{dy}{dt} + 5t \, y = \sin\brak{t} \quad \text{with } y\brak{1} = 2\pi.
\end{align*}

There exists a unique solution for this differential equation when $ t$  belongs to the interval

\hfill{[GATE 2017]} \begin{enumerate}
    \begin{multicols}{2}
        \item $\brak{-2, 2}$
        \item $\brak{-10, 10}$
        \item $\brak{-10, 2}$
        \item $\brak{0, 10}$
    \end{multicols}
\end{enumerate}
\item Consider the line integral 
\begin{align*}
I = \int_C \brak{x^2 + iy} \, dz,
\end{align*}
where $ z = x + iy $. The line $ C $ is shown in the figure below.

\begin{center}
    
\resizebox{0.2\textwidth}{!}{%
\begin{circuitikz}
\tikzstyle{every node}=[font=\LARGE]
\draw  (6.25,10.5) circle (1.75cm);
\draw  (6.25,10.5) circle (0.5cm);
\draw  (6,11.75) rectangle (6.5,11.5);
\draw  (7.5,10.75) rectangle (7.75,10);
\draw  (6,9.5) rectangle (6.5,9.25);
\draw  (4.75,10.75) rectangle (5,10);
\end{circuitikz}
}%

\end{center}

The value of $ I $ is

\hfill{[GATE 2017]} \begin{enumerate}
    \begin{multicols}{2}
        \item $\dfrac{1}{2} i$\\
        \item  $\dfrac{2}{3} i$
        \item $\dfrac{3}{4} i$\\
        \item $\dfrac{4}{5} i$
    \end{multicols}
\end{enumerate}
\item Two passive two-port networks are connected in cascade as shown in the figure. A voltage source is connected at port 1.

\begin{center}
  
\resizebox{0.2\textwidth}{!}{%
\begin{circuitikz}
\tikzstyle{every node}=[font=\LARGE]
\draw  (6.25,10.5) circle (1.75cm);
\draw  (6.25,10.5) circle (0.5cm);
\draw  (5.75,11.75) rectangle (6.75,11.5);
\draw  (7.25,10.5) rectangle (7.75,10.25);
\draw  (5.75,9.5) rectangle (6.75,9.25);
\draw  (4.75,10.5) rectangle (5.25,10.25);
\end{circuitikz}
}%

\end{center}

Given
\begin{align*}
    V_1 &= A_1 V_2 + B_1 I_2\\ I_1 &= C_1 V_2 + D_1 I_2\\
    V_2 &= A_2 V_3 + B_2 I_3 \\ I_2 &= C_2 V_3 + D_2 I_3
\end{align*}
    



where  $A_1, B_1, C_1, D_1, A_2, B_2, C_2$ and $ D_2$ are the generalized circuit constants. If the Thevenin equivalent circuit at port 3 consists of a voltage source  $V_T$ and an impedance $ Z_T$, connected in series, then

\hfill{[GATE 2017]} \begin{enumerate}
    \begin{multicols}{2}
        \item  $V_T = \dfrac{V_1}{A_1 + A_2}, \quad Z_T = \dfrac{A_1 B_2 + B_1 D_2}{A_1 A_2}$\\
        \item  $V_T = \dfrac{V_1}{A_1 A_2 + B_1 C_2}, \quad Z_T = \dfrac{A_1 B_2 + B_1 D_2}{A_1 A_2 + B_1 C_2} $
        \item$ V_T = \dfrac{V_1}{A_1 + A_2}, \quad Z_T = \dfrac{A_1 A_2 + B_1 C_2}{A_1 A_2 + B_1 D_2}$\\
        \item $ V_T = \dfrac{V_1}{A_1 A_2 + B_1 C_2}, \quad Z_T = \dfrac{A_1 A_2 + B_1 C_2}{A_1 A_2 + B_1 C_2}$
    \end{multicols}
\end{enumerate}

\item Let a causal LTI system be characterized by the following differential equation, with initial rest condition
\begin{align*}
   \frac{d^2 y}{dt^2} + 7 \frac{dy}{dt} + 10 y\brak{t} = 4x\brak{t} + 5 \frac{dx\brak{t}}{dt} 
\end{align*}

where $x\brak{t}$ and  $y\brak{t}$ are the input and output respectively. The impulse response of the system is $ u\brak{t}$ is the unit step function

\hfill{[GATE 2017]} \begin{enumerate}
    \begin{multicols}{2}
        \item  $2e^{-2t} u\brak{t} - 7e^{-5t} u\brak{t} $
        \item  $-2e^{-2t} u\brak{t} + 7e^{-5t} u\brak{t} $
        \item $ 7e^{-2t} u\brak{t} - 2e^{-5t} u\brak{t} $
        \item $-7e^{-2t} u\brak{t} + 2e^{-5t} u\brak{t} $
    \end{multicols}
\end{enumerate}
\item Let the signal
\begin{align*}
  x\brak{t} = \sum_{k=-\infty}^{+\infty} \brak{-1}^k \delta \brak{ t - \frac{k}{2000} }
\end{align*}

be passed through an LTI system with frequency response $H\brak{\omega}$, as given in the figure below.
\begin{figure}[!ht]
\centering
\resizebox{0.3\textwidth}{!}{%
\begin{circuitikz}
\tikzstyle{every node}=[font=\LARGE]
\draw [->, >=Stealth] (3.75,5) -- (3.75,11);
\draw [->, >=Stealth] (2.25,6.75) -- (8,6.75);
\draw[short] (3,5) .. controls (4.5,6.5) and (6,9.8) .. (7.75,9.5);
\node [font=\normalsize] at (2.75,10.75) {$C_m$};
\node [font=\normalsize] at (7.5,6.25) {$\alpha$};
\end{circuitikz}
}%

\end{figure}



The Fourier series representation of the output is given as\\

\hfill{[GATE 2017]} \begin{enumerate}
 
        \item $ 4000 + 4000 \cos\brak{2000 \pi t} + 4000 \cos\brak{4000 \pi t} $   \item$ 2000 + 2000 \cos\brak{2000 \pi t} + 2000 \cos\brak{4000 \pi t}$
        \item $ 2000 \cos\brak{2000 \pi t} $
       \item $ 4000 \cos\brak{2000 \pi t}$\\
    
\end{enumerate}

\item In the system whose signal flow graph is shown in the figure,$ U_1\brak{s}$ and  $U_2\brak{s}$ are inputs. The transfer function $\dfrac{Y\brak{s}}{U_1\brak{s}}$  is
\hfill{[GATE 2017]} 
\begin{figure}[!ht]
\centering
\resizebox{0.5\textwidth}{!}{%
\begin{circuitikz}
\tikzstyle{every node}=[font=\large]
\draw [line width=0.2pt, ->, >=Stealth] (3.75,7.5) -- (3.75,11.75);
\draw [line width=0.6pt, ->, >=Stealth] (3.75,8) -- (15,8);
\draw [line width=0.9pt, ->, >=Stealth] (1.5,10.5) -- (2.75,10.5);
\draw [line width=0.9pt, ->, >=Stealth] (1.5,10) -- (2.75,10);
\draw [line width=0.9pt, ->, >=Stealth] (1.5,9.5) -- (2.75,9.5);
\draw [line width=0.9pt, ->, >=Stealth] (1.5,9) -- (2.75,9);
\draw [line width=0.9pt, ->, >=Stealth] (1.5,8) -- (2.75,8);
\draw [line width=0.9pt, ->, >=Stealth] (1.5,8.5) -- (2.75,8.5);
\draw [line width=0.9pt, ->, >=Stealth] (8.75,10.5) -- (10.5,10.5);
\draw [line width=0.9pt, ->, >=Stealth] (8.75,10) -- (10.46,10);
\draw [line width=0.9pt, ->, >=Stealth] (8.75,9.5) -- (10.42,9.5);
\draw [line width=0.9pt, ->, >=Stealth] (8.75,9) -- (10.3,9);
\draw [line width=0.9pt] (8.75,8) -- (8.75,10.75);

\draw [line width=0.9pt, ->, >=Stealth] (8.75,8.5) -- (10.1,8.5);
\draw [line width=0.9pt, short] (3.75,8) .. controls (3.5,10) and (8.75,10.25) .. (12,10.5);
\draw [line width=0.9pt, short] (8.75,8) .. controls (10.5,7.75) and (10.5,9) .. (10.5,10.75);
\draw [line width=0.9pt, short] (10,10) -- (9.75,10);
\node [font=\large] at (4,12.25) {$y$};
\node [font=\large] at (14.25,7.25) {$x$};
\node [font=\large] at (11,9) {$U\brak{y}$};
\node [font=\large] at  (2.25,11.25){$U_{\infty}$};
\end{circuitikz}
}%

\end{figure}
\begin{enumerate}
    \begin{multicols}{2}
        \item $\dfrac{k_1}{JL s^2 + JR s + k_1 k_2}$
        \item  $\dfrac{k_1}{JL s^2 - JR s - k_1 k_2}$
        \item  $\dfrac{k_1 - U_2 \brak{R + sL}}{JL s^2 + \brak{JR - U_2 L} s + k_1 k_2 - U_2 R}$
        \item  $\dfrac{k_1 - U_2 \brak{sL - R}}{JL s^2 - \brak{JR + U_2 L} s - k_1 k_2 + U_2 R}$
    \end{multicols}
\end{enumerate}
\item The transfer function of the system is given by:

\begin{align*}
x\brak{t} &= \myvec{ 1 & 2 \\ 2 & 0} + \myvec {1 \\ 2 } u\brak{t}  \\
y\brak{t} &= \myvec{ 1 & 0} x\brak{t}
\end{align*}
The options for the transfer function are:\hfill{[GATE 2017]} 
\begin{multicols}{2}
\begin{enumerate}
    \item $\dfrac{\brak{s + 2}}{s^2 - 2s - 2}$\\
    \item  $\dfrac{\brak{s - 2}}{s^2 + s - 4}$
    \item  $\dfrac{\brak{s + 4}}{s^2 + s - 4}$\\
    \item  $\dfrac{\brak{s + 4}}{s^2 - s - 4}$
\end{enumerate}     
\end{multicols}

\item The load shown in the figure is supplied by a 400 $V$ (line-to-line), 3-phase source RYB sequence. The load is balanced and inductive, drawing 3464 $VA$. When the switch $S$ is in position $ N $, the three wattmeters $W_1 $,  $W_2$, and $ W_3 $ read 577.35 $W$ each. If the switch is moved to position $Y$, the readings of the wattmeters in watts will be:
\hfill{[GATE 2017]} 

\begin{center}
\resizebox{0.4\textwidth}{!}{%
\begin{circuitikz}
\tikzstyle{every node}=[font=\LARGE]
\draw [->, >=Stealth] (1.75,8) -- (11.25,8);
\draw [->, >=Stealth] (6.25,7.5) -- (6.25,11);
\draw (2.25,10) to[short] (4.25,8);
\draw (4.25,8) to[short] (6.25,10);
\draw (6.25,10) to[short] (8.25,8);
\draw (8.25,8) to[short] (10.25,10);
\node at (4.25,8) [circ] {};
\node at (6.25,10) [circ] {};
\node at (8.25,8) [circ] {};
\node at (10,8) [circ] {};
\node at (2.25,8) [circ] {};
\node [font=\LARGE] at (8.25,7.5) {2};
\node [font=\LARGE] at (10,7.5) {4};
\node [font=\LARGE] at (2.25,7.5) {-4};
\node [font=\LARGE] at (4.25,7.5) {-2};
\node [font=\LARGE] at (6.5,7.75) {0};
\node [font=\LARGE] at (5.75,10) {2};
\node [font=\LARGE] at (6,11.5) {$y$};
\node [font=\LARGE] at (11,7.25) {$x$};
\end{circuitikz}
}%
\end{center}
\item The approximate transfer characteristic for the circuit shown below with an ideal operational amplifier and diode is as follows:

\begin{figure}[!ht]
\centering
\resizebox{0.35\textwidth}{!}{%
\begin{circuitikz}
\tikzstyle{every node}=[font=\normalsize]
\draw [ line width=0.9pt](7.25,5.25) node[op amp,scale=1, yscale=-1 ] (opamp2) {};
\draw [ line width=0.9pt](opamp2.+) to[short] (5.75,5.75);
\draw [ line width=0.9pt] (opamp2.-) to[short] (5.75,4.75);
\draw [ line width=0.9pt](8.45,5.25) to[short](8.75,5.25);
\draw [ line width=0.9pt](8.75,5.25) to[D] (8.75,3);
\draw [ line width=0.9pt](5.75,4.75) to[short] (5.75,3.5);
\draw [ line width=0.9pt](5.75,3) to[short] (10,3);
\draw [ line width=0.9pt](5.75,3) to[short] (5.75,4);
\draw [ line width=0.9pt](9.5,3) to[R] (9.5,1.25);
\draw [line width=0.9pt](9.5,1.5) to (9.5,1.25) node[ground]{};
\draw [ line width=0.9pt](5.75,5.75) to[short] (5.2,5.75);
\node [font=\normalsize] at (4.85,5.75) {$V_{in}$};
\draw [ line width=0.9pt](7,4.25) to[short] (7,4.5);
\draw [ line width=0.9pt](7,5.85) to[short] (7,6.25);
\draw [ line width=0.9pt](7,4.25) to[short] (7,4.67);
\node [font=\normalsize] at (7,6.5) {$V_{SS}$};
\node [font=\normalsize] at (7,4.1) {$V_{SS}$};
\node [font=\normalsize] at (9.5,4.25) {$D$};
\node [font=\normalsize] at (10.5,3) {$V_{o}$};
\node [font=\normalsize] at (10,2.25) {$R$};
\end{circuitikz}
}%

\end{figure}
\hfill{[GATE 2017]}
\begin{multicols}{2}
     \begin{enumerate}
    \item 
\resizebox{0.2\textwidth}{!}{%
\begin{circuitikz}
\tikzstyle{every node}=[font=\normalsize]
\draw [->, >=Stealth] (2.5,5.25) -- (5.75,5.25);
\draw [->, >=Stealth] (2.5,5.25) -- (2.5,8.5);
\draw [line width=0.2pt, short] (3.25,5.25) .. controls (4.25,5.75) and (4.75,6.25) .. (5,7.25);
\node [font=\normalsize, rotate around={90:(0,0)}] at (2.25,6.75) {Response};
\node [font=\normalsize] at (4,5) {Time};
\end{circuitikz}
}%

 
     \item 
\resizebox{0.25\textwidth}{!}{%
\begin{circuitikz}
\tikzstyle{every node}=[font=\LARGE]
\draw [->, >=Stealth] (2.5,8.25) -- (2.5,11.75);
\draw [->, >=Stealth] (2.5,8.25) -- (6.25,8.25);
\node [font=\normalsize] at (2.5,12) {$C_{L}$};
\node [font=\normalsize] at (6.25,8.75) {$C_{D}$};
\node [font=\normalsize] at (2.25,8) {$O$};
\draw [short] (2.5,10.75) .. controls (5.25,9.75) and (5.75,9.5) .. (2.5,8.25);
\end{circuitikz}
}%
  \item 

\resizebox{0.2\textwidth}{!}{%
\begin{circuitikz}
\tikzstyle{every node}=[font=\normalsize]
\draw [ line width=0.8pt](1.25,9) to[short] (5,9);
\draw [ line width=0.8pt](3,9) to[short] (3,10.75);
\draw [ line width=0.8pt](3,9) to[short] (1.5,10.5);
\node [font=\normalsize] at (3,11) {$V_o$};
\node [font=\normalsize] at (5.25,8.75) {$V_{in}$};
\draw [ line width=0.8pt](3,9) to[short] (4.5,10.5);
\end{circuitikz}
}%
  \item 
\resizebox{0.2\textwidth}{!}{%
\begin{circuitikz}
\tikzstyle{every node}=[font=\normalsize]
\draw [ line width=0.8pt](1.25,9) to[short] (5,9);
\draw [ line width=0.8pt](3,9) to[short] (3,10.75);
\node [font=\normalsize] at (3,11) {$V_o$};
\node [font=\normalsize] at (5.25,8.75) {$V_{in}$};
\draw [ line width=0.8pt](1.75,10.25) to[short] (4.25,10.25);
\end{circuitikz}
}%
\end{enumerate}
\end{multicols}

   

    

\item The output expression for the Karnaugh map shown below is
\begin{align*}
  \begin{array}{|c|c|c|c|c|}\hline
    AB \backslash CD & 00 & 01 & 11 & 10 \\
    \hline
    00 & 0 & 0 & 0 & 0 \\\hline
    01 & 1 & 0 & 1 & 1 \\ \hline
    11 & 1 & 0 & 1 & 1 \\ \hline
    10 & 0 & 0 & 0 & 0 \\ \hline
\end{array}  
\end{align*}


\hfill{[GATE 2017]}
\begin{multicols}{4}
\begin{enumerate}
    \item $ \overline{B} \overline{D} + BCD $
    \item $ \overline{B} D + AB $
    \item $ \overline{B} D + ABC $
    \item $ B \overline{D} + ABC $
\end{enumerate}
\end{multicols}
\item A load is supplied by a 230 $V$, 50 $Hz$ source. The active power $ P $ and the reactive power $ Q $ consumed by the load are such that $1 \text{ kW} \leq P \leq 2 \text{ kW}$ and $ 1 \text{ kVAR} \leq Q \leq 2 \text{ kVAR}$. A capacitor connected across the load for power factor correction generates 1 $kVAR$ reactive power. The worst case power factor after power factor correction is
\hfill{[GATE 2017]} 
\begin{multicols}{4}
 \begin{enumerate}
    \item 0.447 lag
    \item 0.707 lag
    \item 0.894 lag
    \item 1
\end{enumerate}
\end{multicols}

\item The logical gate implemented using the circuit shown below where $ V_1$ and $ V_2 $ are inputs with (0 $V$ as digital 0 and 5 $V$ as digital 1) and $ V_{\text{OUT}} $ is the output, is \\ \newpage
\begin{figure}[!ht]
\centering
\begin{circuitikz}[scale=0.85, transform shape] 
\tikzstyle{every node}=[font=\normalsize]
\draw [ line width=0.9pt](6.25,10.5) to[R] (6.25,8.5);
\draw [ line width=0.9pt](6.25,8.5) to[short] (11.25,8.5);
\draw [line width=0.9pt](10.5,6) to[Tnpn, transistors/scale=1.19] (10.5,8.5);
\draw [ line width=0.9pt](9.75,7.25) to[R] (8,7.25);
\draw [line width=0.9pt](10.5,6.25) to (10.5,6) node[ground]{};
\draw [line width=0.9pt](6.25,6.25) to[Tnpn, transistors/scale=1.19] (6.25,8.75);
\draw [ line width=0.9pt](5.5,7.5) to[R] (3.75,7.5);
\draw [line width=0.9pt](6.25,6.75) to (6.25,6.5) node[ground]{};
\draw [ line width=0.9pt](5.75,10.5) to[short] (6.75,10.5);
\node [font=\normalsize] at (8.85,7.75) {1k$\Omega$};
\node [font=\normalsize] at (7,9.75) {1K$\Omega$};
\node [font=\normalsize] at (4.5,8) {1k$\Omega$};
\node [font=\normalsize] at (6.5,10.75) {5 $V$};
\node [font=\normalsize] at (3.5,7.75) {$V_1$};
\node [font=\normalsize] at (10.5,8.75) {$V_{o}$};
\node [font=\normalsize] at (7.75,7.5) {$V_2$};
\node [font=\normalsize] at (6.25,7.5) {$Q_1$};
\node [font=\normalsize] at (10.5,7.25) {$Q_2$};
\end{circuitikz}

\end{figure}


\hfill{[GATE 2017]}
\item The input voltage $V_{DC}$ of the buck-boost converter shown below varies from 32 $V$ to 72 $V$. Assume that all components are ideal, inductor current is continuous, and output voltage is ripple-free. The range of duty ratio $D$ of the converter for which the magnitude of the steady-state output voltage remains constant at 48 $V$ is
\hfill{[GATE 2017]}
\begin{figure}[!ht]
\centering
\resizebox{0.45\textwidth}{!}{%
\begin{circuitikz}
\tikzstyle{every node}=[font=\normalsize]

\draw [short] (4,10.75) .. controls (6.25,11) and (6,9.5) .. (7.5,10.5);
\draw [dashed] (8.25,10.75) .. controls (8.75,11.25) and (10.5,10.25) .. (11.25,10);

\draw (3.25,10.75) to[short] (3.25,11.25);
\draw (2.75,11.25) to[short] (3.25,11.25);

\draw (2.75,8.5) to[short] (3.25,8.5);
\draw (3.25,9) to[short] (3.25,8.5);
\draw [short] (4,9) .. controls (6,8.5) and (6.25,10) .. (7.5,9.25);
\draw (3.25,10.75) to[short] (4,10.75);
\draw (3.25,9) to[short] (4,9);
\draw [short] (7.5,10.5) -- (8.25,10.75);
\draw [short] (7.5,9.25) -- (8.25,9);
\draw [dashed] (8.25,9) .. controls (9,8.25) and (9.75,9.25) .. (11,9.5);
\draw[dashed] (8.25,10.75) -- (9,10);
\draw[dashed] (8.3,10.77) -- (9.2,9.9);

\draw[dashed] (8.25,9) -- (9,10);
\draw[dashed] (8.3,8.97) -- (9.12,10);

\draw[dashed] (9,10) -- (10.25,10.3);
\draw[dashed] (9.2,10) -- (10.457,10.3);
\draw[dashed] (9,10) -- (10.32,9.25);
\draw[dashed] (9.1,10) -- (10.42,9.26);

\node [font=\normalsize] at (3,10) {$P_o$};
\node [font=\normalsize] at (3,9.5) {$T_o$};
\node [font=\normalsize] at (4.25,10) {FLOW};
\draw [->, >=Stealth] (3.5,9.75) -- (5.25,9.75);
\node [font=\normalsize] at (6.75,8.75) {NOZZLE-A};
\draw [->, >=Stealth] (8.5,8) -- (8.5,9);
\draw [->, >=Stealth] (9.75,11.75) -- (8.75,10.75);
\draw [->, >=Stealth] (11.3,11.3) -- (10.5,10.25);
\node [font=\normalsize] at (10,12) {OBLIQUE SHOCK};
\node [font=\normalsize] at (11.75,11.5) {EXPANSION FAN};
\node [font=\normalsize] at (8.5,7.5) {NOZZLE EXIT};
\end{circuitikz}
}%

\end{figure}

\begin{multicols}{4}
\begin{enumerate}
    \item $ \dfrac{2}{5} \leq D \leq \dfrac{3}{5} $
    \item $ \dfrac{2}{3} \leq D \leq \dfrac{3}{4} $
    \item $ 0 \leq D \leq 1 $
    \item $ \dfrac{1}{3} \leq D \leq \dfrac{2}{3} $
\end{enumerate}

\end{multicols}




\end{enumerate}


\end{document}
