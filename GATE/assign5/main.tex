\let\negmedspace\undefined
\let\negthickspace\undefined
\documentclass[journal,12pt,onecolumn]{IEEEtran}
\usepackage{cite}
\usepackage{amsmath,amssymb,amsfonts,amsthm}
\usepackage{amsmath}
\usepackage{algorithmic}
\usepackage{graphicx}
\usepackage{textcomp}
\usepackage{xcolor}
\usepackage{txfonts}
\usepackage{listings}
\usepackage{multicol}
\usepackage{enumitem}
\usepackage{mathtools}
\usepackage{gensymb}
\usepackage{comment}
\usepackage[breaklinks=true]{hyperref}
\usepackage{tkz-euclide} 
\usepackage{listings}
\usepackage{gvv}                                        
\usepackage[latin1]{inputenc}                                
\usepackage{color}                                            
\usepackage{array}                                            
\usepackage{longtable}                                       
\usepackage{calc}                                             
\usepackage{multirow}                                         
\usepackage{hhline}                                           
\usepackage{ifthen}                                           
\usepackage{lscape}
\usepackage{tabularx}
\usepackage{array}
\usepackage{float}
\usepackage{tikz}
\usepackage{multicol}
\usetikzlibrary{patterns}
\newtheorem{theorem}{Theorem}[section]
\newtheorem{problem}{Problem}
\newtheorem{proposition}{Proposition}[section]
\newtheorem{lemma}{Lemma}[section]
\newtheorem{corollary}[theorem]{Corollary}
\newtheorem{example}{Example}[section]
\newtheorem{definition}[problem]{Definition}
\newcommand{\BEQA}{\begin{eqnarray}}
\newcommand{\EEQA}{\end{eqnarray}}
\newcommand{\define}{\stackrel{\triangle}{=}}
\theoremstyle{remark}
\newtheorem{rem}{Remark}

\begin{document}

\bibliographystyle{IEEEtran}
\vspace{3cm}

\title{2014-ME}
\author{EE24BTECH11020 -  Ellanti Rohith}
\maketitle

\renewcommand{\thefigure}{\theenumi}
\renewcommand{\thetable}{\theenumi}
\begin{enumerate}
    \item The state of stress at a point is given by $\sigma_x = -6$ $MPa$, $\sigma_y = 4$ $MPa$, and $\tau_{xy} = -8$ $MPa$. The maximum tensile stress (in $MPa$) at the point is \underline{\hspace{2cm}} \hfill{[GATE 2014]}
\\

    \item A block $R$ of mass 100 $kg$ is placed on a block $S$ of mass 150 $kg$ as shown in the figure. Block $R$ is tied to the wall by a massless and inextensible string $PQ$. If the coefficient of static friction for all surfaces is 0.4, the minimum force $F$ (in $kN$) needed to move the block $S$ is \hfill{[GATE 2014]}\\
    \begin{center}
    \begin{figure}[!ht]
\centering
\resizebox{0.45\textwidth}{!}{%
\begin{circuitikz}
\tikzstyle{every node}=[font=\normalsize]

\draw [short] (4,10.75) .. controls (6.25,11) and (6,9.5) .. (7.5,10.5);
\draw [dashed] (8.25,10.75) .. controls (8.75,11.25) and (10.5,10.25) .. (11.25,10);

\draw (3.25,10.75) to[short] (3.25,11.25);
\draw (2.75,11.25) to[short] (3.25,11.25);

\draw (2.75,8.5) to[short] (3.25,8.5);
\draw (3.25,9) to[short] (3.25,8.5);
\draw [short] (4,9) .. controls (6,8.5) and (6.25,10) .. (7.5,9.25);
\draw (3.25,10.75) to[short] (4,10.75);
\draw (3.25,9) to[short] (4,9);
\draw [short] (7.5,10.5) -- (8.25,10.75);
\draw [short] (7.5,9.25) -- (8.25,9);
\draw [dashed] (8.25,9) .. controls (9,8.25) and (9.75,9.25) .. (11,9.5);
\draw[dashed] (8.25,10.75) -- (9,10);
\draw[dashed] (8.3,10.77) -- (9.2,9.9);

\draw[dashed] (8.25,9) -- (9,10);
\draw[dashed] (8.3,8.97) -- (9.12,10);

\draw[dashed] (9,10) -- (10.25,10.3);
\draw[dashed] (9.2,10) -- (10.457,10.3);
\draw[dashed] (9,10) -- (10.32,9.25);
\draw[dashed] (9.1,10) -- (10.42,9.26);

\node [font=\normalsize] at (3,10) {$P_o$};
\node [font=\normalsize] at (3,9.5) {$T_o$};
\node [font=\normalsize] at (4.25,10) {FLOW};
\draw [->, >=Stealth] (3.5,9.75) -- (5.25,9.75);
\node [font=\normalsize] at (6.75,8.75) {NOZZLE-A};
\draw [->, >=Stealth] (8.5,8) -- (8.5,9);
\draw [->, >=Stealth] (9.75,11.75) -- (8.75,10.75);
\draw [->, >=Stealth] (11.3,11.3) -- (10.5,10.25);
\node [font=\normalsize] at (10,12) {OBLIQUE SHOCK};
\node [font=\normalsize] at (11.75,11.5) {EXPANSION FAN};
\node [font=\normalsize] at (8.5,7.5) {NOZZLE EXIT};
\end{circuitikz}
}%

\end{figure}


    \end{center}

    \begin{multicols}{4}
    \begin{enumerate}
         \item 0.69
        \item 0.88
        \item 0.98
        \item 1.37
    
    \end{enumerate}
       
    \end{multicols}

    \item A pair of spur gears with module 5 $mm$ and a center distance of 450 $mm$ is used for a speed reduction of 5:1. The number of teeth on pinion is \underline{\hspace{2cm}} \hfill{[GATE 2014]}\\

   
    \item Consider a cantilever beam, having negligible mass and uniform flexural rigidity, with length 0.01 $m$. The frequency of vibration of the beam, with a 0.5 $kg$ mass attached at the free tip, is 100 $Hz$. The flexural rigidity (in N$\cdot$m$^2$) of the beam is \underline{\hspace{2cm}} \hfill{[GATE 2014]}\\

    \item An ideal water jet with volume flow rate of 0.05 m$^3$/s strikes a flat plate placed normal to its path and exerts a force of 1000 $N$. Considering the density of water as 1000 $kg/m$ $^3$, the diameter (in $mm$) of the water jet is \underline{\hspace{2cm}} \hfill{[GATE 2014]}\\
 

    \item A block weighing 200 $N$ is in contact with a level plane whose coefficients of static and kinetic friction are 0.4 and 0.2, respectively. The block is acted upon by a horizontal force (in newton) $P = 10t$, where $t$ denotes the time in seconds. The velocity (in $m/s$) of the block attained after 10 seconds is \underline{\hspace{2cm}} \hfill{[GATE 2014]}\\

      \item A slider crank mechanism has slider mass of 10 $kg$, stroke of 0.2 $m$ and rotates with a uniform angular velocity of 10 $rad/s$. The primary inertia forces of the slider are partially balanced by a revolving mass of 6 $kg$ at the crank, placed at a distance equal to crank radius. Neglect the mass of connecting rod and crank. When the crank angle (with respect to slider axis) is 30$\degree$, the unbalanced force (in newton) normal to the slider axis is\underline{\hspace{2cm}} \hfill{[GATE 2014]}
\\
 \item An offset slider-crank mechanism is shown in the figure at an instant. Conventionally, the Quick  Return Ratio (QRR) is considered to be greater than one. The value of QRR is 

 
\begin{centering}
  \begin{figure}[!ht]
\centering
\begin{circuitikz}[scale=0.85, transform shape] 
\tikzstyle{every node}=[font=\normalsize]
\draw [ line width=0.9pt](6.25,10.5) to[R] (6.25,8.5);
\draw [ line width=0.9pt](6.25,8.5) to[short] (11.25,8.5);
\draw [line width=0.9pt](10.5,6) to[Tnpn, transistors/scale=1.19] (10.5,8.5);
\draw [ line width=0.9pt](9.75,7.25) to[R] (8,7.25);
\draw [line width=0.9pt](10.5,6.25) to (10.5,6) node[ground]{};
\draw [line width=0.9pt](6.25,6.25) to[Tnpn, transistors/scale=1.19] (6.25,8.75);
\draw [ line width=0.9pt](5.5,7.5) to[R] (3.75,7.5);
\draw [line width=0.9pt](6.25,6.75) to (6.25,6.5) node[ground]{};
\draw [ line width=0.9pt](5.75,10.5) to[short] (6.75,10.5);
\node [font=\normalsize] at (8.85,7.75) {1k$\Omega$};
\node [font=\normalsize] at (7,9.75) {1K$\Omega$};
\node [font=\normalsize] at (4.5,8) {1k$\Omega$};
\node [font=\normalsize] at (6.5,10.75) {5 $V$};
\node [font=\normalsize] at (3.5,7.75) {$V_1$};
\node [font=\normalsize] at (10.5,8.75) {$V_{o}$};
\node [font=\normalsize] at (7.75,7.5) {$V_2$};
\node [font=\normalsize] at (6.25,7.5) {$Q_1$};
\node [font=\normalsize] at (10.5,7.25) {$Q_2$};
\end{circuitikz}

\end{figure}


   
\end{centering}\hfill{[GATE 2014]}
 \\

  
\item A rigid uniform rod $AB$ of length $L$ and mass $m$ is hinged at $C$ such that $AC = L/3$, $CB = 2L/3$. Ends $A$ and $B$ are supported by springs of spring constant $k$. The natural frequency of the system is given by \\


\begin{centering}
     \begin{figure}[!ht]
\centering
\resizebox{0.35\textwidth}{!}{%
\begin{circuitikz}
\tikzstyle{every node}=[font=\normalsize]
\draw [ line width=0.9pt](7.25,5.25) node[op amp,scale=1, yscale=-1 ] (opamp2) {};
\draw [ line width=0.9pt](opamp2.+) to[short] (5.75,5.75);
\draw [ line width=0.9pt] (opamp2.-) to[short] (5.75,4.75);
\draw [ line width=0.9pt](8.45,5.25) to[short](8.75,5.25);
\draw [ line width=0.9pt](8.75,5.25) to[D] (8.75,3);
\draw [ line width=0.9pt](5.75,4.75) to[short] (5.75,3.5);
\draw [ line width=0.9pt](5.75,3) to[short] (10,3);
\draw [ line width=0.9pt](5.75,3) to[short] (5.75,4);
\draw [ line width=0.9pt](9.5,3) to[R] (9.5,1.25);
\draw [line width=0.9pt](9.5,1.5) to (9.5,1.25) node[ground]{};
\draw [ line width=0.9pt](5.75,5.75) to[short] (5.2,5.75);
\node [font=\normalsize] at (4.85,5.75) {$V_{in}$};
\draw [ line width=0.9pt](7,4.25) to[short] (7,4.5);
\draw [ line width=0.9pt](7,5.85) to[short] (7,6.25);
\draw [ line width=0.9pt](7,4.25) to[short] (7,4.67);
\node [font=\normalsize] at (7,6.5) {$V_{SS}$};
\node [font=\normalsize] at (7,4.1) {$V_{SS}$};
\node [font=\normalsize] at (9.5,4.25) {$D$};
\node [font=\normalsize] at (10.5,3) {$V_{o}$};
\node [font=\normalsize] at (10,2.25) {$R$};
\end{circuitikz}
}%

\end{figure}



\end{centering}    
   
\hfill{[GATE 2014]}
    \begin{multicols}{4}
    \begin{enumerate}
        \item $\sqrt{\frac{k}{2m}}$
        \item $\sqrt{\frac{k}{m}}$
        \item $\sqrt{\frac{2k}{m}}$
        \item $\sqrt{\frac{5k}{m}}$
    \end{enumerate}
    \end{multicols}
    \item A hydrodynamic journal bearing is subject to 2000 $N$ load at a rotational speed of 2000 rpm. Both bearing bore diameter and length are 40 $mm$. If radial clearance is 20 $\mu m$ and bearing is lubricated with an oil having viscosity 0.03 $Pa.s$, the Sommerfeld number of the bearing is \underline{\hspace{2cm}}\hfill{[GATE 2014]}
   \\
    

    \item A 200 $mm$ long, stress-free rod at room temperature is held between two immovable rigid walls. The temperature of the rod is uniformly raised by 250$\degree C$. If the Young's modulus and coefficient of thermal expansion are 200 $GPa$ and $1 \times 10^{-5}/ \degree C$, respectively, the magnitude of the longitudinal stress (in $MPa$) developed in the rod is \underline{\hspace{2cm}}\hfill{[GATE 2014]}
  \\
  
\item 1.5 kg of water is in saturated liquid state at 2 bar ($v_f = 0.001061$ m$^3 /kg$, $u_f = 504.0$ $kJ/kg$, $h_f = 505$ $kJ/kg$). Heat is added in a constant pressure process till the temperature of water reaches 400$\degree C$ ($v = 1.5493$ m$^3 /kg$, $u = 2967.0$ $kJ/kg$, $h = 3277.0$ /, $kJ/kg$). The heat added (in $kJ$) in the process is \underline{\hspace{2cm}}\hfill{[GATE 2014]}
 \\
    
 \item Consider one dimensional steady state heat conduction across a wall (as shown in figure below) of thickness $30 \ \text{mm}$ and thermal conductivity $15 \ \text{W/m.K}$. At $x = 0$, a constant heat flux, $q'' = 1 \times 10^5 \ \text{W/m}^2$ is applied. On the other side of the wall, heat is removed from the wall by convection with a fluid at $25^\circ \text{C}$ and heat transfer coefficient of $250 \ \text{W/m}^2\cdot\text{K}$. The temperature (in $^\circ \text{C}$), at $x = 0$ is \underline{\hspace{2cm}}\hfill{[GATE 2014]}

\begin{center}
  \begin{center}
\resizebox{0.4\textwidth}{!}{%
\begin{circuitikz}
\tikzstyle{every node}=[font=\LARGE]
\draw [->, >=Stealth] (1.75,8) -- (11.25,8);
\draw [->, >=Stealth] (6.25,7.5) -- (6.25,11);
\draw (2.25,10) to[short] (4.25,8);
\draw (4.25,8) to[short] (6.25,10);
\draw (6.25,10) to[short] (8.25,8);
\draw (8.25,8) to[short] (10.25,10);
\node at (4.25,8) [circ] {};
\node at (6.25,10) [circ] {};
\node at (8.25,8) [circ] {};
\node at (10,8) [circ] {};
\node at (2.25,8) [circ] {};
\node [font=\LARGE] at (8.25,7.5) {2};
\node [font=\LARGE] at (10,7.5) {4};
\node [font=\LARGE] at (2.25,7.5) {-4};
\node [font=\LARGE] at (4.25,7.5) {-2};
\node [font=\LARGE] at (6.5,7.75) {0};
\node [font=\LARGE] at (5.75,10) {2};
\node [font=\LARGE] at (6,11.5) {$y$};
\node [font=\LARGE] at (11,7.25) {$x$};
\end{circuitikz}
}%
\end{center}
    \end{center}

 
\end{enumerate}

\end{document}

