\let\negmedspace\undefined
\let\negthickspace\undefined
\documentclass[journal,12pt,onecolumn]{IEEEtran}
\usepackage{cite}
\usepackage{amsmath,amssymb,amsfonts,amsthm}
\usepackage{amsmath}
\usepackage{algorithmic}
\usepackage{graphicx}
\usepackage{textcomp}
\usepackage{xcolor}
\usepackage{txfonts}
\usepackage{listings}
\usepackage{multicol}
\usepackage{enumitem}
\usepackage{mathtools}
\usepackage{gensymb}
\usepackage{comment}
\usepackage[breaklinks=true]{hyperref}
\usepackage{tkz-euclide} 
\usepackage{listings}
\usepackage{gvv}                                        
\usepackage[latin1]{inputenc}                                
\usepackage{color}                                            
\usepackage{array}                                            
\usepackage{longtable}                                       
\usepackage{calc}                                             
\usepackage{multirow}                                         
\usepackage{hhline}                                           
\usepackage{ifthen}                                           
\usepackage{lscape}
\usepackage{tabularx}
\usepackage{array}
\usepackage{float}
\usepackage{tikz}
\usepackage{multicol}
\usepackage{circuitikz}
\usetikzlibrary{patterns}
\newtheorem{theorem}{Theorem}[section]
\newtheorem{problem}{Problem}
\newtheorem{proposition}{Proposition}[section]
\newtheorem{lemma}{Lemma}[section]
\newtheorem{corollary}[theorem]{Corollary}
\newtheorem{example}{Example}[section]
\newtheorem{definition}[problem]{Definition}
\newcommand{\BEQA}{\begin{eqnarray}}
\newcommand{\EEQA}{\end{eqnarray}}
\newcommand{\define}{\stackrel{\triangle}{=}}
\theoremstyle{remark}
\newtheorem{rem}{Remark}

\begin{document}

\bibliographystyle{IEEEtran}
\vspace{3cm}

\title{2019-AE}
\author{EE24BTECH11020 -  Ellanti Rohith}
\maketitle

\renewcommand{\thefigure}{\theenumi}
\renewcommand{\thetable}{\theenumi}
\begin{enumerate}

    \item The dimensions of kinematic viscosity of a fluid (where $L$ is length, $T$ is time) are
    \hfill{[GATE 2019]}
    \begin{multicols}{4}
        \begin{enumerate}
            \item $LT^{-1}$
            \item $L^2T^{-1}$
            \item $LT^{-2}$
            \item $L^2T$
        \end{enumerate}
    \end{multicols}

    \item $\phi\brak{x, y}$ represents the velocity potential of a two-dimensional flow with velocity field $\vec{V} = u\brak{x, y} \hat{i} + v\brak{x, y} \hat{j}$, where $\hat{i}$ and $\hat{j}$ are unit vectors along the $x$ and $y$ axes, respectively. Which of the following is necessarily true?
    \hfill{[GATE 2019]}
    \begin{multicols}{2}
        \begin{enumerate}
            \item $\nabla^2 \phi = 0$
            \item $\nabla \times \vec{V} = 0$
            \item $\nabla \cdot \vec{V} = 0$
            \item $u = -\dfrac{\partial \phi}{\partial y}, \, v = \dfrac{\partial \phi}{\partial x}$
        \end{enumerate}
    \end{multicols}

    \item For a quasi-one-dimensional isentropic supersonic flow through a diverging duct, which of the following is true in the direction of the flow?
    \hfill{[GATE 2019]}

        \begin{enumerate}
            \item Both the Mach number and the static temperature increase.
            \item The Mach number increases and the static temperature decreases.
            \item The Mach number decreases and the static temperature increases.
            \item Both the Mach number and the static temperature decrease.\\
        \end{enumerate}
        
    \item For a NACA2415 airfoil of chord length $c$, which of the following is true?
    \hfill{[GATE 2019]}

        \begin{enumerate}
            \item Maximum camber is located at $0.2c$ from the leading edge.
            \item Maximum thickness is located at $0.15c$ from the leading edge.
            \item Maximum camber is $0.02c$.
            \item Maximum thickness is $0.05c$.\\
        \end{enumerate}

\item When a propeller airplane in ground-roll during take-off experiences headwind, which of the following statements is FALSE?
    \hfill{[GATE 2019]}
    \begin{multicols}{2}
        \begin{enumerate}
            \item The drag on the airplane increases.
            \item The thrust from the propellers decreases.
            \item The wing lift increases.
            \item The ground-roll distance increases.
        \end{enumerate}
    \end{multicols}

    \item Which of the following graphs represents the response of a dynamically unstable airplane?
    \hfill{[GATE 2019]}
    \begin{multicols}{2}
        \begin{enumerate}
        \item 
\resizebox{0.25\textwidth}{!}{%
\begin{circuitikz}
\tikzstyle{every node}=[font=\LARGE]
\draw [->, >=Stealth] (2.5,8.25) -- (2.5,11.75);
\draw [->, >=Stealth] (2.5,8.25) -- (6.25,8.25);
\node [font=\normalsize] at (2.5,12) {$C_{L}$};
\node [font=\normalsize] at (6.25,8.75) {$C_{D}$};
\node [font=\normalsize] at (2.25,8) {$O$};
\draw [short] (2.5,10.75) .. controls (4,7.75) and (4.5,7.25) .. (5.75,10.75);
\end{circuitikz}
}%


        \item 
\resizebox{0.25\textwidth}{!}{%
\begin{circuitikz}
\tikzstyle{every node}=[font=\LARGE]
\draw [->, >=Stealth] (2.5,8.25) -- (2.5,11.75);
\draw [->, >=Stealth] (2.5,8.25) -- (6.25,8.25);
\node [font=\normalsize] at (2.5,12) {$C_{L}$};
\node [font=\normalsize] at (6.25,8.75) {$C_{D}$};
\node [font=\normalsize] at (2.25,8) {$O$};
\draw [short] (2.5,10.75) .. controls (5.25,9.75) and (5.75,9.5) .. (2.5,8.25);
\end{circuitikz}
}%

        \item 
\resizebox{0.25\textwidth}{!}{%
\begin{circuitikz}
\tikzstyle{every node}=[font=\LARGE]
\draw [->, >=Stealth] (2.5,8.25) -- (2.5,11.75);
\draw [->, >=Stealth] (2.5,8.25) -- (6.25,8.25);
\node [font=\normalsize] at (2.5,12) {$C_{L}$};
\node [font=\normalsize] at (6.25,8.75) {$C_{D}$};
\node [font=\normalsize] at (2.25,8) {$O$};
\draw [short] (2.5,8.25) .. controls (3.5,11.75) and (4.5,11.25) .. (5.25,8.25);
\end{circuitikz}
}%

        \item 
\resizebox{0.15\textwidth}{!}{%
\begin{circuitikz}
\tikzstyle{every node}=[font=\normalsize]

\filldraw[fill=gray!30, draw=black] 
    (5,8.25) -- (6.5,9) -- (8,7.75) -- (6.5,5.75) -- cycle;

\node at (6.5,8) {\textbf{R}};

\end{circuitikz}
}%
        \end{enumerate}
    \end{multicols}

\item The propulsive efficiency of a ramjet engine is lower than that of a low bypass turbofan engine operating under the same conditions and producing the same thrust, primarily because the ramjet engine
    \hfill{[GATE 2019]}
    \begin{multicols}{2}
        \begin{enumerate}
            \item has larger kinetic energy lost in the exhaust jet.
            \item has lower thrust power.
            \item is not self-starting.
            \item has higher thrust to weight ratio.
        \end{enumerate}
    \end{multicols}

\item While flying at Mach 2.0, 11  $km$   altitude and producing the same thrust, what is the correct order from the lowest thrust specific fuel consumption $\brak{tsfc}$ to the highest $ tsfc $ ?
    \hfill{[GATE 2019]}
    \begin{multicols}{2}
        \begin{enumerate}
            \item Turbofan, Ramjet, Turbojet
            \item Turbofan, Turbojet, Ramjet
            \item Ramjet, Turbojet, Turbofan
            \item Turbojet, Turbofan, Ramjet
        \end{enumerate}
    \end{multicols}

    \item For a single stage subsonic compressor, which of the following statements about the highest possible compressor pressure ratio (CPR) is correct?
    \hfill{[GATE 2019]}
   
        \begin{enumerate}
            \item CPR of an axial compressor $>$ CPR of centrifugal compressor.
            \item CPR of an axial compressor $<$ CPR of centrifugal compressor.
            \item CPR of an axial compressor = CPR of centrifugal compressor.
            \item CPR of any value can be attained with either an axial or a centrifugal compressor.
        \end{enumerate}
 

    \item For a beam subjected to a transverse shear load through its shear center,
    \hfill{[GATE 2019]}
    
        \begin{enumerate}
            \item the twist per unit length is zero.
            \item the shear stress is uniform throughout the cross-section.
            \item the bending stresses in the cross section are zero.
            \item the shear strain is zero at the shear center.
        \end{enumerate}
    


  \item A function $f\brak{x}$ is defined by $f\brak{x} = \dfrac{1}{2}\brak{x + |x|}$. The value of \begin{align*}
      \int_{-1}^{1} f\brak{x} \, dx
  \end{align*} is \underline{\hspace{2cm} } (round off to 1 decimal place). \hfill{[GATE 2019]}

    \item The value of the following limit is \underline{\hspace{2cm}}(round off to 2 decimal places).
    
    \begin{align*}
    \lim_{\theta \to 0} \dfrac{\theta - \sin \theta}{\theta^3}
    \end{align*}
    \hfill{[GATE 2019]}
    
    \item To simulate the aerodynamic forces on a cylinder of 1 $m$diameter due to a uniform air flow of 1 $m/s$ at standard temperature and pressure (STP), low-speed wind tunnel experiments at STP are conducted on a 0.1 $m$ diameter cylinder. The free stream air speed in the wind tunnel experiments should be \underline{\hspace{2cm}} $m/s$ (round off to the nearest integer).
    \hfill{[GATE 2019]}
\end{enumerate}


\end{document}