%iffalse
\let\negmedspace\undefined
\let\negthickspace\undefined
\documentclass[journal,12pt,twocolumn]{IEEEtran}
\usepackage{cite}
\usepackage{amsmath,amssymb,amsfonts,amsthm}
\usepackage{algorithmic}
\usepackage{graphicx}
\usepackage{textcomp}
\usepackage{xcolor}
\usepackage{txfonts}
\usepackage{listings}
\usepackage{enumitem}
\usepackage{mathtools}
\usepackage{gensymb}
\usepackage{comment}
\usepackage[breaklinks=true]{hyperref}
\usepackage{tkz-euclide} 
\usepackage{listings}
\usepackage{gvv}                                        
\def\inputGnumericTable{}                                 
\usepackage[latin1]{inputenc}                                
\usepackage{color}                                            
\usepackage{array}                                            
\usepackage{longtable}                                       
\usepackage{calc}                                             
\usepackage{multirow}                                         
\usepackage{hhline}                                           
\usepackage{ifthen}                                           
\usepackage{lscape}
\usepackage{multicol}


\newtheorem{theorem}{Theorem}[section]
\newtheorem{problem}{Problem}
\newtheorem{proposition}{Proposition}[section]
\newtheorem{lemma}{Lemma}[section]
\newtheorem{corollary}[theorem]{Corollary}
\newtheorem{example}{Example}[section]
\newtheorem{definition}[problem]{Definition}
\newcommand{\BEQA}{\begin{eqnarray}}
\newcommand{\EEQA}{\end{eqnarray}}
\newcommand{\define}{\stackrel{\triangle}{=}}
\theoremstyle{remark}
\newtheorem{rem}{Remark}
\begin{document}

\bibliographystyle{IEEEtran}
\vspace{3cm}

\title{Assignment-1}
\author{EE24BTECH11020 - Ellanti Rohith% <-this % stops a space
}
\maketitle
\newpage
\bigskip

\renewcommand{\thefigure}{\theenumi}
\renewcommand{\thetable}{\theenumi}
\begin{enumerate}
	\item {If, for positive integer n,the quadratic equation, $x(x+1)+(x+1)(x+2)+....(x+{n-1})(x+n)=10n$  has two consecutive integral solutions, then n is equal to:     \hfill{[JEE M 2017]} \begin{multicols}{2} 
\begin{enumerate}
    \item {11}
    \item{12}
    \columnbreak
    \item {9} 
    \item{10}\end{enumerate} 
    \end{multicols}}


 \item{For any three positive real number a,b and c $9(25a^2+b^2)+ 25(c^2-3ac) = 15b(3a+c)$. Then:\\ \hfill{[JEE M 2017]}\begin{multicols}{2}
	 \begin{enumerate}\itemsep.5em
  \item{a,b and c are in A.P}
  \item{b,c and a are in G.P}
  \columnbreak
  \item{b,c and a are in A.P}
  \item{a,b and c are in G.P}
  \end{enumerate}
  \end{multicols}}
  \item{Let a,b,c$\epsilon$ R. If $f(x)=ax^2+bx+c$ is such that a+b+c=3 and f(x+y)=f(x)+f(y) $\forall$ x,y $\epsilon$ R, then $\sum _{n=1}^{10}$  f(n)  is  equal  to\hfill{[2017]}\\\\(a)255 \hspace{2cm}(b)330\\(c)165 \hspace{2cm}(d)190} 
  \item{Let $a_{1},a_{2},a_{3},...,a_{49}$ be an A.P such that $\sum_{k=0}^{12} a_{4k+1}=416$ and $a_{9}+a_{43}=66$. If $a_{1}^2+ a_{2}^{2}+...+a_{17}^{2}=140m$, then m is equal to:\hfill{[JEE M 2018]}\begin{enumerate}
  \begin{multicols}{2
  }\item {68}\item {34}\columnbreak\item{33}  \item{66}
  \end{multicols}
  \end{enumerate}}
  \item{Let A be the sum of the frst 20 terms and B be the sum of the first 40 terms of the series \[1^{2} +2\cdot2^{2}+3^{2}+2\cdot4^{2}+5^{2}+2\cdot6^{2}+...\] If B-2A=100$\lambda$, then $\lambda$ can be\hfill{[JEE M 2018]\begin{enumerate}
  \begin{multicols}{2}
  \item {248} \item{464}\columnbreak
  \item{496}
  \item{232}
  \end{multicols}
  \end{enumerate}}
  \item{If a, b and c be three distinct real numbers in GP. and $a+b+c=xb$, then x cannot be:\hfill{[JEE M 2019]}
\begin{enumerate}
  \begin{multicols}{2}
  \item {-2} \item{4}\columnbreak
  \item{-3}
  \item{2}
  \end{multicols}
  \end{enumerate}}
  \item {Let $a_{1},a_{2},......a_{30}$ be an A.P, $S=\sum_{i=1}^{30} and T=\sum_{i=2}^{15}a_{(2i-1)}$. If $a_{5}=27$
	  nd S-2T=75, then $a_{10}$ is equal to \hfill{[JEE M 2019]}
\begin{enumerate}
  \begin{multicols}{2}
  \item {52} \item{57}\columnbreak
  \item{47}
  \item{42}
  \end{multicols}
  \end{enumerate}}
\item{Three circles of radii a, b, c ($a<b<c$) touch each other externally. If they have x-axis as a common tangent, then:\hfill{[JEE M 2019]}\begin{enumerate} \itemsep.5em	
  
  \item {$\frac{1}{\sqrt{a}}=\frac{1}{\sqrt{b}}+\frac{1}{\sqrt{c}}$} 
  \item {$\frac{1}{\sqrt{b}}=\frac{1}{\sqrt{c}}+\frac{1}{\sqrt{a}}$} 
  \item{a,b and c are in A.P}
  \item{${\sqrt{a}}={\sqrt{b}}+{\sqrt{c}}$}
  \end{enumerate}} 
  \item{Let the sum of the first n terms of a non-constant A.P.,$a_{1},a_{2},a_{3}$... be 50n + $\dfrac{n(n-7)}{2}A$
	  where A is a constant. If d is the common difference of this A.P, then ordered pair (d,$a_{50}$) is equal to\hfill{[JEE M 2019]}} \vspace{.5em}\begin{enumerate}
		  \itemsep.5em  \item {(50, 50+46A)} \item{(50, 50+45A)}
  \item{(A, 50+45A)}
  \item{(A, 50+46A)}
  \end{enumerate}}
  \end{enumerate}
  
\end{document}

